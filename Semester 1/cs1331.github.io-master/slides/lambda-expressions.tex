\documentclass{beamer}

\newcommand{\lesson}{Lambda Expressions}


\newcommand{\course}{Introduction to Object-Oriented Programming}
\subject{\course}
\title[\lesson]{\course}
\subtitle{\lesson}

\author[CS 1331]
{Christopher Simpkins \\\texttt{chris.simpkins@gatech.edu}}
\institute[Georgia Tech]

\date[]{}

\newcommand{\link}[2]{\href{#1}{\textcolor{blue}{\underline{#2}}}}
\newcommand{\code}{http://cs1331.gatech.edu/code}

\usepackage{colortbl}

% If you have a file called "university-logo-filename.xxx", where xxx
% is a graphic format that can be processed by latex or pdflatex,
% resp., then you can add a logo as follows:

% \pgfdeclareimage[width=0.6in]{coc-logo}{cc_2012_logo}
% \logo{\pgfuseimage{coc-logo}}

\mode<presentation>
{
  \usetheme{Berlin}
  \useoutertheme{infolines}

  % or ...

 \setbeamercovered{transparent}
  % or whatever (possibly just delete it)
}

\usepackage{tikz}
% Optional PGF libraries
\usepackage{pgflibraryarrows}
\usepackage{pgflibrarysnakes}
\usepackage{pgfplots}
\usepackage{fancybox}
\usepackage{listings}
\usepackage{hyperref}
\hypersetup{colorlinks=true,urlcolor=blue}
\usepackage[english]{babel}
% or whatever

\usepackage[latin1]{inputenc}
% or whatever

\usepackage{times}
\usepackage[T1]{fontenc}
% Or whatever. Note that the encoding and the font should match. If T1
% does not look nice, try deleting the line with the fontenc.


\usepackage{listings}

% "define" Scala
\lstdefinelanguage{scala}{
  morekeywords={abstract,case,catch,class,def,%
    do,else,extends,false,final,finally,%
    for,if,implicit,import,match,mixin,%
    new,null,object,override,package,%
    private,protected,requires,return,sealed,%
    super,this,throw,trait,true,try,%
    type,val,var,while,with,yield},
  otherkeywords={=>,<-,<\%,<:,>:,\#,@},
  sensitive=true,
  morecomment=[l]{//},
  morecomment=[n]{/*}{*/},
  morestring=[b]",
  morestring=[b]',
  morestring=[b]""",
}

\usepackage{color}
\definecolor{dkgreen}{rgb}{0,0.6,0}
\definecolor{gray}{rgb}{0.5,0.5,0.5}
\definecolor{mauve}{rgb}{0.58,0,0.82}

% Default settings for code listings
\lstset{frame=tb,
  language=scala,
  aboveskip=2mm,
  belowskip=2mm,
  showstringspaces=false,
  columns=flexible,
  basicstyle={\scriptsize\ttfamily},
  numbers=none,
  numberstyle=\tiny\color{gray},
  keywordstyle=\color{blue},
  commentstyle=\color{dkgreen},
  stringstyle=\color{mauve},
  frame=single,
  breaklines=true,
  breakatwhitespace=true,
  keepspaces=true
  %tabsize=3
}


% \beamerdefaultoverlayspecification{<+->}

\begin{document}

\begin{frame}
  \titlepage
\end{frame}

%------------------------------------------------------------------------
\begin{frame}[fragile]{Inner Classes}

Recall from \link{\code/collections/super-troopers/SortTroopers.java}{SortTroopers.java} the {\tt MustacheComparator} class:

\begin{lstlisting}[language=Java]
static class MustacheComparator implements Comparator<Trooper> {

    public int compare(Trooper a, Trooper b) {
        if (a.hasMustache() && !b.hasMustache()) {
            return 1;
        } else if (b.hasMustache() && !a.hasMustache()) {
            return -1;
        } else {
            return a.getName().compareTo(b.getName());
        }
    }
}
\end{lstlisting}

which we can use just like any other named class:

\begin{lstlisting}[language=Java]
Collections.sort(troopers, new MustacheComparator());
\end{lstlisting}


\end{frame}
%------------------------------------------------------------------------


%------------------------------------------------------------------------
\begin{frame}[fragile]{Anonymous Inner Classes}

We can subclass {\tt Comprator} and make an instance of the subclass at the same time using an {\it anonymous inner class}.  Here's a mustache comparator as an inner class:

\begin{lstlisting}[language=Java]
Collections.sort(troopers, new Comparator<Trooper>() {
    public int compare(Trooper a, Trooper b) {
        if (a.hasMustache() && !b.hasMustache()) {
            return 1;
        } else if (b.hasMustache() && !a.hasMustache()) {
            return -1;
        } else {
            return a.getName().compareTo(b.getName());
        }
    }
});
\end{lstlisting}

The general syntax for defining an anonymous inner class is
\[
\text{new } SuperType<TypeArgument>\text{() } \{ class\_body \}
\]

\end{frame}
%------------------------------------------------------------------------

%------------------------------------------------------------------------
\begin{frame}[fragile]{Functional Interfaces}

Any interface with a single abstract method is a functional interface.  For example, {\tt Comparator} is a functional interface:

\begin{lstlisting}[language=Java]
public interface Comparator<T> {
    int compare(T o1, T o2);
}
\end{lstlisting}

As in the previous examples, we only need to implement the single abstract method {\tt compare} to make an instantiable class that implements {\tt Comparator}.\\

\vspace{.05in}

Note that there's an optional {\tt @FunctionalInterface} annotation that is similar to the {\tt @Override} annotation.  Tagging an interface as a {\tt @FunctionalInterface} prompts the compiler to check that the interface indeed contains a single abstract method and includes a statement in the interface's Javadoc that the interface is a functional interface.

\end{frame}
%------------------------------------------------------------------------

%------------------------------------------------------------------------
\begin{frame}[fragile]{Lambda Expressions}

A {\it lambda expression} is a syntactic shortcut for defining the single abstract method of a funtional interface and instantiating an anonymous class that implements the interface.  The general syntax is

\[
(T_1 \hspace{.05in} p_1, ..., T_n \hspace{.05in} p_n) \hspace{.05in} \text{->} \hspace{.05in} \{ method\_body \}
\]
Where
\begin{itemize}
\item $T_1, ..., T_n$ are types and
\item $p_1, ..., p_n$ are parameter names
\end{itemize}
just like in method definitions.\\

\vspace{.05in}

If $method\_body$ is a single expression, the curly braces can be omitted.


\end{frame}
%------------------------------------------------------------------------

%------------------------------------------------------------------------
\begin{frame}[fragile]{MustacheComparator as a Lambda Expression}

Here's our mustache comparator from  \link{\code/collections/super-troopers/LambdaTroopers.java}{LambdaTroopers.java} as a lambda expression:

\begin{lstlisting}[language=Java]
Collections.sort(troopers, (Trooper a, Trooper b) -> {
    if (a.hasMustache() && !b.hasMustache()) {
        return 1;
    } else if (b.hasMustache() && !a.hasMustache()) {
        return -1;
    } else {
        return a.getName().compareTo(b.getName());
    }
});
\end{lstlisting}

\begin{itemize}
\item Because {\tt Collections.sort(List<T> l, Comparator<T> c)} takes a {\tt Comparator<T>}, we way that {\tt Comparator<T>} is the {\it target type} of the lambda expression passed to the {\tt sort} method.
\item The lambda expression creates an instance of an anonymous class that implements {\tt Comparator<Trooper>} and passes this instance to {\tt sort}
\end{itemize}

\end{frame}
%------------------------------------------------------------------------

%------------------------------------------------------------------------
\begin{frame}[fragile]{Target Types}
\vspace{-.05in}
\begin{lstlisting}[language=Java]
static interface Bar {
    int compare(Trooper a, Trooper b);
}
static void foo(Bar b) { ... }
\end{lstlisting}
\vspace{-.05in}
Given the {\tt Bar} interface, the call:
\vspace{-.05in}
\begin{lstlisting}[language=Java]
foo((Trooper a, Trooper b) -> {
    if (a.hasMustache() && !b.hasMustache()) {
        return 1;
    } else if (b.hasMustache() && !a.hasMustache()) {
        return -1;
    } else {
        return a.getName().compareTo(b.getName());
    }
});
\end{lstlisting}
\vspace{-.05in}
creates an instance of the {\tt Bar} interface using the same lambda expression.
\vspace{-.05in}
\begin{itemize}
\item The type of object instantiated by a lambda expression is determined by the {\it target type} of the call in which the lambda expression appears.
\end{itemize}


\end{frame}
%------------------------------------------------------------------------



%------------------------------------------------------------------------
\begin{frame}[fragile]{Revisiting WordCount}

Remember the rank comparator we defined for {\tt WordCount}:
\begin{lstlisting}[language=Java]
public class WordCount {

    private Map<String, Integer> wordCounts;

    public Set<String> getWordsRanked() {
        Comparator<String> rankComparator = new Comparator<String>() {
            public int compare(String k1, String k2) {
                return wordCounts.get(k2) - wordCounts.get(k1);
            }
        };
        TreeSet<String> rankedWords = new TreeSet<>(rankComparator);
        rankedWords.addAll(wordCounts.keySet());
        return rankedWords;
    }
\end{lstlisting}


\end{frame}
%------------------------------------------------------------------------

%------------------------------------------------------------------------
\begin{frame}[fragile]{WordCount's Comparator as a Lambda Expression}

We can replace the anonymous inner class definition with a lambda expression:

\begin{lstlisting}[language=Java]
public Set<String> getWordsRanked() {
    Comparator<String> rankComparator =
        (String k1, String k2) -> wordCounts.get(k2) - wordCounts.get(k1);
    TreeSet<String> rankedWords = new TreeSet<>(rankComparator);
    rankedWords.addAll(wordCounts.keySet());
    return rankedWords;
}
\end{lstlisting}

Notice that since the body of the lambda expression is a single expression, we leave off the curly braces and {\tt return} keyword.

\end{frame}
%------------------------------------------------------------------------

%------------------------------------------------------------------------
\begin{frame}[fragile]{Free and Bound Variables}

\begin{lstlisting}[language=Java]
public class WordCount {
  private Map<String, Integer> wordCounts;

  public Set<String> getWordsRanked() {
    Comparator<String> rankComparator =
     (String k1, String k2) -> wordCounts.get(k2)-wordCounts.get(k1);
    TreeSet<String> rankedWords = new TreeSet<>(rankComparator);
    rankedWords.addAll(wordCounts.keySet());
    return rankedWords;
  }
\end{lstlisting}

In {\tt rankComparator}:

\begin{itemize}
\item {\tt k1} and {\tt k2} are {\it bound variables}.  They are defined in the parameter list or body of the lambda expression.
\item {\tt wordCounts} is a {\it free variable}.  It is defined outside the lambda expression. Free variables must be {\it effectively final}.
\end{itemize}

We say that the lambda expression {\it captures} the {\tt wordCount} variable.  Such lambda expressions are called {\it closures}.

\end{frame}
%------------------------------------------------------------------------

%------------------------------------------------------------------------
\begin{frame}[fragile]{Method References}

\begin{itemize}
\item A lambda expression is a compact notation for specifying the implementation of the abstract method in a functional interface.
\item A method reference is a compact notation for a lambda expression that supplies the implementation of the abstract method in a functional interface from a compatible named method that has already been defined.
\end{itemize}

If a method already exists that fits the specification for a parameter that could take a lambda expression as an argument, you can use a method reference instead of a lambda expression.
\end{frame}
%------------------------------------------------------------------------

%------------------------------------------------------------------------
\begin{frame}[fragile]{Method References Example}
\vspace{-.05in}
Say we have a functional interface whose abstract method takes a single {\tt Object} and returns {\tt void}:
\vspace{-.05in}
\begin{lstlisting}[language=Java]
public interface Foo {
    void bar(Object o);
}
\end{lstlisting}
\vspace{-.05in}
and a method that takes an instance of an object implementing this functional interface as a paramter:
\vspace{-.05in}
\begin{lstlisting}[language=Java]
void doo(Foo f) {
    f.bar("baz");
}
\end{lstlisting}
\vspace{-.05in}
We can supply a method reference to any method that is {\it lambda equivalent} to the {\tt bar} method above (same parameter list and return type):
\vspace{-.05in}
\begin{lstlisting}[language=Java]
doo(System.out::println);
\end{lstlisting}
\vspace{-.05in}
 which is equivalent to:
\vspace{-.05in}
\begin{lstlisting}[language=Java]
doo(x -> System.out.println(x));
\end{lstlisting}

\end{frame}
%------------------------------------------------------------------------

%------------------------------------------------------------------------
\begin{frame}[fragile]{Method References}

Three kinds of method references:
\begin{itemize}
\item {\it object}::{\it instanceMethod} - like {\tt x -> object.instanceMethod(x)}

\item {\it Class}::{\it staticMethod} - like {\tt x -> Class.staticMethod(x)}
\begin{lstlisting}[language=Java]
someList.forEach(System.out::println);
\end{lstlisting}

\item {\it Class}::{\it instanceMethod} - like {\tt (x, y) -> x.instanceMethod(y)}
\begin{lstlisting}[language=Java]
Comparator<Trooper> byName = Comparator.comparing(Trooper::getName);
\end{lstlisting}

\end{itemize}

Let's see some examples ...

\end{frame}
%------------------------------------------------------------------------

%------------------------------------------------------------------------
\begin{frame}[fragile]{Functional(ish) Composition}

Remember how our mustache comparator ordered by mustache, then by name?\\

With lambdas we can make that even more concise and clear:

\begin{lstlisting}[language=Java]
        Comparator<Trooper> byMustacheThenName =
            Comparator.comparing(Trooper::hasMustache)
            .thenComparing(Trooper::getName);
        Collections.sort(troopers, byMustacheThenName);
\end{lstlisting}

Look at the \link{http://doocs.oracle.com/javase/8/docs/api/java/util/Comparator.html}{Comparator} API for details on these methods.

\end{frame}
%------------------------------------------------------------------------


% %------------------------------------------------------------------------
% \begin{frame}[fragile]{}

% \begin{itemize}
% \item
% \end{itemize}

% \begin{lstlisting}[language=Java]

% \end{lstlisting}


% \end{frame}
% %------------------------------------------------------------------------

\end{document}
