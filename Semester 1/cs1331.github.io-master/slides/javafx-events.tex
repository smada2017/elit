\documentclass{beamer}

\newcommand{\lesson}{JavaFX Events}


\newcommand{\course}{Introduction to Object-Oriented Programming}
\subject{\course}
\title[\lesson]{\course}
\subtitle{\lesson}

\author[CS 1331]
{Christopher Simpkins \\\texttt{chris.simpkins@gatech.edu}}
\institute[Georgia Tech]

\date[]{}

\newcommand{\link}[2]{\href{#1}{\textcolor{blue}{\underline{#2}}}}
\newcommand{\code}{http://cs1331.gatech.edu/code}

\usepackage{colortbl}

% If you have a file called "university-logo-filename.xxx", where xxx
% is a graphic format that can be processed by latex or pdflatex,
% resp., then you can add a logo as follows:

% \pgfdeclareimage[width=0.6in]{coc-logo}{cc_2012_logo}
% \logo{\pgfuseimage{coc-logo}}

\mode<presentation>
{
  \usetheme{Berlin}
  \useoutertheme{infolines}

  % or ...

 \setbeamercovered{transparent}
  % or whatever (possibly just delete it)
}

\usepackage{tikz}
% Optional PGF libraries
\usepackage{pgflibraryarrows}
\usepackage{pgflibrarysnakes}
\usepackage{pgfplots}
\usepackage{fancybox}
\usepackage{listings}
\usepackage{hyperref}
\hypersetup{colorlinks=true,urlcolor=blue}
\usepackage[english]{babel}
% or whatever

\usepackage[latin1]{inputenc}
% or whatever

\usepackage{times}
\usepackage[T1]{fontenc}
% Or whatever. Note that the encoding and the font should match. If T1
% does not look nice, try deleting the line with the fontenc.


\usepackage{listings}

% "define" Scala
\lstdefinelanguage{scala}{
  morekeywords={abstract,case,catch,class,def,%
    do,else,extends,false,final,finally,%
    for,if,implicit,import,match,mixin,%
    new,null,object,override,package,%
    private,protected,requires,return,sealed,%
    super,this,throw,trait,true,try,%
    type,val,var,while,with,yield},
  otherkeywords={=>,<-,<\%,<:,>:,\#,@},
  sensitive=true,
  morecomment=[l]{//},
  morecomment=[n]{/*}{*/},
  morestring=[b]",
  morestring=[b]',
  morestring=[b]""",
}

\usepackage{color}
\definecolor{dkgreen}{rgb}{0,0.6,0}
\definecolor{gray}{rgb}{0.5,0.5,0.5}
\definecolor{mauve}{rgb}{0.58,0,0.82}

% Default settings for code listings
\lstset{frame=tb,
  language=scala,
  aboveskip=2mm,
  belowskip=2mm,
  showstringspaces=false,
  columns=flexible,
  basicstyle={\scriptsize\ttfamily},
  numbers=none,
  numberstyle=\tiny\color{gray},
  keywordstyle=\color{blue},
  commentstyle=\color{dkgreen},
  stringstyle=\color{mauve},
  frame=single,
  breaklines=true,
  breakatwhitespace=true,
  keepspaces=true
  %tabsize=3
}


% \beamerdefaultoverlayspecification{<+->}

\begin{document}

\begin{frame}
  \titlepage
\end{frame}

%------------------------------------------------------------------------
\begin{frame}[fragile]{Outline}


\begin{itemize}
\item Hello, JavaFX!
\item Event-driven programming
\item Hello, buttons!
\item The observer pattern
\end{itemize}


\end{frame}
%------------------------------------------------------------------------

%------------------------------------------------------------------------
\begin{frame}[fragile]{Hello, JavaFX!}


Here's a minimal JavaFX program:
\begin{lstlisting}[language=Java]
import javafx.application.Application;
import javafx.scene.control.Label;
import javafx.scene.text.Font;
import javafx.scene.Scene;
import javafx.stage.Stage;

public class HelloJfx extends Application {
    public void start(Stage stage) {
        Label message = new Label("Hello, JavaFX!");
        message.setFont(new Font(100));
        stage.setScene(new Scene(message));
        stage.setTitle("Hello");
        stage.show();
    }
}
\end{lstlisting}

See \link{\code/javafx/HelloJfx.java}{HelloJfx.java} and the API documentation for \link{http://docs.oracle.com/javase/8/javafx/api/}{JavaFX}

\end{frame}
%------------------------------------------------------------------------

%------------------------------------------------------------------------
\begin{frame}[fragile]{{\tt javafx.application.Application}}
\vspace{-.06in}
JavaFX programs include one class that extends \link{http://docs.oracle.com/javase/8/javafx/api/javafx/application/Application.html}{\tt Application}, analogous to having a single class with a {\tt main} method for console programs.  When running an {\tt Application} the JavaFX run-time:
\vspace{-.03in}
\begin{enumerate}
  \item Constructs an instance of the specified Application class
\item Calls the {\tt init()} method for application initialization - {\bf don't construct a {\tt Stage} or {\tt Scene} in {\tt init()}}
\item Calls the {\tt start(javafx.stage.Stage)} method
\item Waits for the application to finish, which happens when either of the following occur:
\begin{itemize}
    \item the application calls \link{http://docs.oracle.com/javase/8/javafx/api/javafx/application/Platform.html}{\tt Platform.exit()}
    \item the last window has been closed and the {\tt implicitExit} attribute on {\tt Platform} is {\tt true}, which is the default
\end{itemize}
\item Calls the {\tt stop()} method - release resources obtained in {\tt init()}
\end{enumerate}

{\tt init()} and {\tt stop()} have default do-nothing implementaitons.  To write a simple JavaFX program you create a subclass of {\tt Application} and put GUI start-up code in the {\tt start(javafx.stage.Stage)} method.

\end{frame}
%------------------------------------------------------------------------

%------------------------------------------------------------------------
\begin{frame}[fragile]{Setting the Stage}

The JavaFX {\tt Stage} class is the top level JavaFX container.  A primary stage for your application is constructed by the JavaFX run-time and passed to your application in the {\tt start(javafx.stage.Stage)} method:

\begin{lstlisting}[language=Java]
    @Override public void start(Stage stage) {
        Scene root = ...
        stage.setScene(new Scene(root));
        stage.setTitle("Hello");
        stage.show();
    }
\end{lstlisting}

You construct a {\tt Stage} for each window in your application, e.g., for dialogs and pop-ups, and add visual components to the stage using a scene graph.

\end{frame}
%------------------------------------------------------------------------

%------------------------------------------------------------------------
\begin{frame}[fragile]{Setting the Scene}

The JavaFX \link{http://docs.oracle.com/javase/8/javafx/api/javafx/scene/Scene.html}{\tt Scene} class is the container for all content in a scene graph.  Every {\tt Scene} has a root node, which may have children (which is why it's called a scene {\it graph}).  In \link{\code/javafx/HelloJfx.java}{HelloJfx.java} we simply used a {\tt Label} control as the root node:

\begin{lstlisting}[language=Java]
    @Override public void start(Stage stage) {
        Label message = new Label("Hello, JavaFX!");
        stage.setScene(new Scene(message));
        stage.setTitle("Hello");
        stage.show();
    }
\end{lstlisting}

A typical application will use as its root node a
\begin{itemize}
  \item \link{http://docs.oracle.com/javase/8/javafx/api/javafx/scene/Group.html}{Group}, typically used for graphics and animation components,
\item a \link{http://docs.oracle.com/javase/8/javafx/api/javafx/scene/layout/Region.html}{Region} class for nodes that can be resized and styled with CSS (like UI controls or layout panes), or
\item a \link{http://docs.oracle.com/javase/8/javafx/api/javafx/scene/layout/Pane.html}{Pane} subclass for laying out children according to some layout policy.
\end{itemize}

\end{frame}
%------------------------------------------------------------------------


%------------------------------------------------------------------------
\begin{frame}[fragile]{Event-Driven Programming}


So far we've done structured sequential programming where the order of execution is controlled by the programmer.  GUIs use event-driven programming:
\begin{itemize}
\item User is presented with options.
\item User actions (and other actions) fire events.
\item Event handlers execute in response to events.
\item Order of execution is controlled by the order of events, which the programmer does not know in advance.
\end{itemize}


\end{frame}
%------------------------------------------------------------------------

%------------------------------------------------------------------------
\begin{frame}[fragile]{Hello, buttons!}

\vspace{-.1in}
\begin{lstlisting}[language=Java]
public class HelloJfxButtons extends Application {

    int count = 0;

    @Override public void start(Stage stage) {
        Label counterLabel = new Label("Count: 0");
        Button incButton = new Button("Increment Count");
        incButton.setOnAction(event ->
            { counterLabel.setText("Count: " + (++count)); });

        VBox root = new VBox();
        root.getChildren().addAll(counterLabel, incButton);
        Scene scene = new Scene(root);
        stage.setScene(scene);
        stage.setTitle("Hello");
        stage.show();
    }
}
\end{lstlisting}
Note that \link{\code/javafx/HelloJfxButtons.java}{HelloJfxButtons.java} uses a \link{http://docs.oracle.com/javase/8/javafx/api/javafx/scene/layout/VBox.html}{VBox} as its layout manager.  We'll have more to say about layout in a future lecture.

\end{frame}
%------------------------------------------------------------------------

%------------------------------------------------------------------------
\begin{frame}[fragile]{Top-Level GUI Program Recipe}

The \link{\code/javafx/HelloJfxButtons.java}{HelloJfxButtons.java} example demonstrates a simple recipe for JavaFX GUI programs:
\begin{enumerate}
\item Create UI controls
\begin{lstlisting}[language=Java]
        Label counterLabel = new Label("Count: 0");
        Button incButton = new Button("Increment Count");
        incButton.setOnAction(event ->
            { counterLabel.setText("Count: " + (++count)); });
\end{lstlisting}
\item Add UI controls to a parent node (Group, Region, or Pane)
\begin{lstlisting}[language=Java]
        VBox root = new VBox();
        root.getChildren().addAll(counterLabel, incButton);
        Scene scene = new Scene(root);
\end{lstlisting}
\item Set the stage with the scene graph and show it
\begin{lstlisting}[language=Java]
        stage.setScene(scene);
        stage.setTitle("Hello");
        stage.show();
\end{lstlisting}
\end{enumerate}

\end{frame}
%------------------------------------------------------------------------


%------------------------------------------------------------------------
\begin{frame}[fragile]{The Observer Pattern in JavaFX}


Three particpants in the observer pattern:
\begin{itemize}
\item An event publisher that fires events
\item An event object that represent the event
\item Event handlers that subscribe to event publishers and receive event objects
\end{itemize}

Practically speaking, firing an event means calling a method on event listeners.  Let's look at a concrete example.




\end{frame}
%------------------------------------------------------------------------

%------------------------------------------------------------------------
\begin{frame}[fragile]{An Event Publisher: {\tt javafx.scene.control.Button}}


In {\tt HelloJfxButtons.java} we set up an increment button like this:
\begin{lstlisting}[language=Java]
         Button incButton = new Button("Increment Count");
         incButton.setOnAction(event ->
            { counterLabel.setText("Count: " + (++count)); });
\end{lstlisting}

\begin{itemize}
\item {\tt Button}'s {\tt setOnAction} method takes an object that implements the {\tt javafx.event.EventHandler<ActionEvent>} interface.
\begin{itemize}
\item {\tt javafx.event.EventHandler<ActionEvent>} has one abstract method: {\tt void handle(ActionEvent event)}
\end{itemize}
\item When the button is pressed, the {\tt void handle(ActionEvent event)} is invoked on the object passed to {\tt setOnAction}, in this case an anonymous inner class that implements the {\tt EventHandler} interface, which was instantiated by a lambda expression.
\end{itemize}


\end{frame}
%------------------------------------------------------------------------

%------------------------------------------------------------------------
\begin{frame}[fragile]{{\tt javafx.event.EventHandler<T>}}


\begin{lstlisting}[language=Java]
@FunctionalInterface
public interface EventHandler<T extends Event>
        extends EventListener {

    /**
     * Invoked when a specific event of the type for which this
     * handler is registered happens.
     *
     * @param event the event which occurred
     */
    void handle(T event);

}
\end{lstlisting}

\begin{itemize}
\item {\tt java.util.EventListener} is a tagging interface that all event listener interfaces must extend (an implementation detail you don't need to worry about).
\end{itemize}


\end{frame}
%------------------------------------------------------------------------

%------------------------------------------------------------------------
\begin{frame}[fragile]{Our Button Event Handler}

Consider this alternative syntax for setting an event handler for our button (just to reinforce that a lambda expression is just syntax sugar for creating anonymous inner class instances of functional interfaces):
\begin{lstlisting}[language=Java]
        incButton.setOnAction(new EventHandler<ActionEvent>() {
            public void handle(ActionEvent e) {
                counterLabel.setText("Count: " + (++count));
            }
        });
\end{lstlisting}

\begin{itemize}
\item Our {\tt EventHandler} captures the {\tt counterLabel} local variable (which is effectively final) and the {\tt count} instance variable
\item When its {\tt handle} method is called, it updates the count and (re-) sets the text on {\tt counterLabel} with the new count
\end{itemize}
Three objects cooperating: a {\tt Button}, a {\tt Label}, and an {\tt EventHandler}  to tie them together.

\end{frame}
%------------------------------------------------------------------------

% %------------------------------------------------------------------------
% \begin{frame}[fragile]{Properties}
%
% \begin{itemize}
% \item
% \end{itemize}
%
% \begin{lstlisting}[language=Java]
%
% \end{lstlisting}
%
%
% \end{frame}
% %------------------------------------------------------------------------
%
% %------------------------------------------------------------------------
% \begin{frame}[fragile]{Binding}
%
% \begin{itemize}
% \item
% \end{itemize}
%
% \begin{lstlisting}[language=Java]
%
% \end{lstlisting}
%
%
% \end{frame}
% %------------------------------------------------------------------------


%------------------------------------------------------------------------
\begin{frame}[fragile]{Closing Thoughts}


\begin{itemize}
\item Event-driven GUI programming requires a shift in thinking.  Putting the user in control means you have to work harder to
\begin{itemize}
  \item handle order dependencies, e.g., by disabling buttons until certain actions are taken, and
  \item guide the user, e.g., by following UI guidelines to maximize familiarity.
\end{itemize}
\item Notice how the JavaFX framework contains classes and interfaces that you extend and implement - OOP in action.
\end{itemize}


\end{frame}
%------------------------------------------------------------------------

% %------------------------------------------------------------------------
% \begin{frame}[fragile]{}

% \begin{itemize}
% \item
% \end{itemize}

% \begin{lstlisting}[language=Java]

% \end{lstlisting}


% \end{frame}
% %------------------------------------------------------------------------

\end{document}
