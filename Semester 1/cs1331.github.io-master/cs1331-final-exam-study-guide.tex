\documentclass[addpoints,9pt]{exam}

\usepackage{verbatim, multicol, tabularx,}
\usepackage{amsmath,amsthm, amssymb, latexsym, listings, qtree}

\lstset{frame=tb,
  language=Java,
  aboveskip=1mm,
  belowskip=0mm,
  showstringspaces=false,
  columns=flexible,
  basicstyle={\ttfamily},
  numbers=none,
  frame=single,
  breaklines=true,
  breakatwhitespace=true
}

\textwidth = 6.5 in
\textheight = 9 in
\oddsidemargin = 0.0 in
\evensidemargin = 0.0 in
\topmargin = -0.25 in
\headheight = 0.0 in
\headsep = 0.0 in
\parskip = 0.0 in
\parindent = 0.0 in



\title{CS 1331 Final Exam}
\date{Study Guide}
\setcounter{page}{0}
\begin{document}


\def\a{& $\blacksquare\blacksquare\blacksquare$ & [ B ] & [ C ] & [ D ] \\}
\def\b{& [ A ] & $\blacksquare\blacksquare\blacksquare$ & [ C ] & [ D ] \\}
\def\c{& [ A ] & [ B ] & $\blacksquare\blacksquare\blacksquare$ & [ D ] \\}
\def\d{& [ A ] & [ B ] & [ C ] & $\blacksquare\blacksquare\blacksquare$ \\}


% \maketitle

\begin{center}
{\LARGE CS 1331 Final Exam}\\
\vspace{.2in}
{\large Study Guide}
\end{center}

\thispagestyle{head}


\runningheader{}
              {\tiny Copyright \textcopyright\ 2016 All rights reserved. Duplication and/or usage for purposes of any kind without permission is strictly forbidden.}
              {}

\footer{Page \thepage\ of \numpages}
              {}
              {Points available: \pointsonpage{\thepage} -
               points lost: \makebox[.5in]{\hrulefill} =
               points earned:  \makebox[.5in]{\hrulefill}.
              Graded by: \makebox[.5in]{\hrulefill}}


\ifprintanswers
\begin{center}
{\LARGE ANSWER KEY}
\end{center}
\else
\fi

\vfill

% Points Table
%\begin{center}
\addpoints
%\gradetable[v][pages]
%\end{center}

% Points Table
Completely fill in the box corresponding to your answer choice for each question.

\ifprintanswers
\begin{multicols}{2}
\begin{tabular}{lcccc}\\
  1. \a
  2. \a
  3. \a
  4. \a
  5. \a
  6. \a
  7. \b
  8. \b
  9. \a
  10. \a
  11. \b
  12. \b
  13. \c
  14. \b
  15. \b
  16. \a
  17. \c
  18. \b
  19. \a
  20. \a
  21. \d
  22. \a
  23. \d
  24. \a
  25. \a
\end{tabular}

\columnbreak

\begin{tabular}{lcccc}\\
  26. \b
  27. \a
  28. \a
  29. \a
  30. \a
  31. \a
  32. \d
  33. \a
  34. \a
  35. \c
  36. \a
  37. \b
  %% 38. \a
  %% 39. \a
  %% 40. \a
  %% 41. \a
  %% 42. \a
  %% 43. \a
  %% 44. \a
  %% 45. \a
  %% 46. \a
  %% 47. \a
  %% 48. \a
  %% 49. \a
  %% 50. \a
\end{tabular}

\end{multicols}

\else

\begin{multicols}{2}

\begin{tabular}{lcccc}\\
  1. & [ A ] & [ B ] & [ C ] & [ D ] \\
  2. & [ A ] & [ B ] & [ C ] & [ D ] \\
  3. & [ A ] & [ B ] & [ C ] & [ D ] \\
  4. & [ A ] & [ B ] & [ C ] & [ D ] \\
  5. & [ A ] & [ B ] & [ C ] & [ D ] \\
  6. & [ A ] & [ B ] & [ C ] & [ D ] \\
  7. & [ A ] & [ B ] & [ C ] & [ D ] \\
  8. & [ A ] & [ B ] & [ C ] & [ D ] \\
  9. & [ A ] & [ B ] & [ C ] & [ D ] \\
  10. & [ A ] & [ B ] & [ C ] & [ D ] \\
  11. & [ A ] & [ B ] & [ C ] & [ D ] \\
  12. & [ A ] & [ B ] & [ C ] & [ D ] \\
  13. & [ A ] & [ B ] & [ C ] & [ D ] \\
  14. & [ A ] & [ B ] & [ C ] & [ D ] \\
  15. & [ A ] & [ B ] & [ C ] & [ D ] \\
  16. & [ A ] & [ B ] & [ C ] & [ D ] \\
  17. & [ A ] & [ B ] & [ C ] & [ D ] \\
  18. & [ A ] & [ B ] & [ C ] & [ D ] \\
  19. & [ A ] & [ B ] & [ C ] & [ D ] \\
  20. & [ A ] & [ B ] & [ C ] & [ D ] \\
  21. & [ A ] & [ B ] & [ C ] & [ D ] \\
  22. & [ A ] & [ B ] & [ C ] & [ D ] \\
  23. & [ A ] & [ B ] & [ C ] & [ D ] \\
  24. & [ A ] & [ B ] & [ C ] & [ D ] \\
  25. & [ A ] & [ B ] & [ C ] & [ D ] \\
\end{tabular}

\columnbreak

\begin{tabular}{lcccc}\\
  26. & [ A ] & [ B ] & [ C ] & [ D ] \\
  27 & [ A ] & [ B ] & [ C ] & [ D ] \\
  28. & [ A ] & [ B ] & [ C ] & [ D ] \\
  29. & [ A ] & [ B ] & [ C ] & [ D ] \\
  30. & [ A ] & [ B ] & [ C ] & [ D ] \\
  31. & [ A ] & [ B ] & [ C ] & [ D ] \\
  32. & [ A ] & [ B ] & [ C ] & [ D ] \\
  33. & [ A ] & [ B ] & [ C ] & [ D ] \\
  34. & [ A ] & [ B ] & [ C ] & [ D ] \\
  35. & [ A ] & [ B ] & [ C ] & [ D ] \\
  36. & [ A ] & [ B ] & [ C ] & [ D ] \\
  37. & [ A ] & [ B ] & [ C ] & [ D ] \\
  %% 38. & [ A ] & [ B ] & [ C ] & [ D ] \\
  %% 39. & [ A ] & [ B ] & [ C ] & [ D ] \\
  %% 40. & [ A ] & [ B ] & [ C ] & [ D ] \\
  %% 41. & [ A ] & [ B ] & [ C ] & [ D ] \\
  %% 42. & [ A ] & [ B ] & [ C ] & [ D ] \\
  %% 43. & [ A ] & [ B ] & [ C ] & [ D ] \\
  %% 44. & [ A ] & [ B ] & [ C ] & [ D ] \\
  %% 45. & [ A ] & [ B ] & [ C ] & [ D ] \\
  %% 46. & [ A ] & [ B ] & [ C ] & [ D ] \\
  %% 47. & [ A ] & [ B ] & [ C ] & [ D ] \\
  %% 48. & [ A ] & [ B ] & [ C ] & [ D ] \\
  %% 49. & [ A ] & [ B ] & [ C ] & [ D ] \\
  %% 50. & [ A ] & [ B ] & [ C ] & [ D ] \\
\end{tabular}


\end{multicols}

\fi

\vspace{.2in}

Number missed: \makebox[.5in]{\hrulefill} Final Score: \makebox[.5in]{\hrulefill}

\newpage

%\normalsize

\pointsinmargin
\bracketedpoints

\marginpointname{}
%%%%%%%%%%%%%%%%%%%%%%%%%%%%%%%%%%%%%%%%%%%%%%%%%%%%%%%%%%%%%%%%%%%%%%%%%%%%

\begin{questions}

\begin{lstlisting}
    public class Kitten {

        private String name = "";

        public Kitten(String name) {
            name = name;
        }

        public String toString() {
            return "Kitten: " + name;
        }

        public boolean equals(Object other) {
            if (this == other) return true;
            if (null == other) return false;
            if (!(other instanceof Kitten)) return false;
            Kitten that = (Kitten) other;
            return this.name.equals(that.name);
        }
    }
\end{lstlisting}
Assume the following statements have been executed:
\begin{lstlisting}
        Object maggie = new Kitten("Maggie");
        Object fiona = new Kitten("Fiona");
        Object fiona2 = new Kitten("Fiona");
\end{lstlisting}

\question[3] What is the value of {\tt maggie}?

\begin{choices}
\correctchoice the address of a {\tt Kitten} object
\choice null
\choice automatically set to 0
\choice undefined
\end{choices}


\question[3] What is printed on the console after the following statement is executed?

\verb@ System.out.println(maggie.toString());@

\begin{choices}
\correctchoice Kitten:
\choice Kitten: null
\choice Kitten: Maggie
\end{choices}

\question[3] What is the value of the expression {\tt fiona.equals(fiona2)}?

\begin{choices}
\correctchoice true
\choice false
\end{choices}

\question[3] What is the value of the expression {\tt fiona.equals(maggie)}?

\begin{choices}
\correctchoice true
\choice false
\end{choices}

\question[3] After executing {\tt Kitten[] kittens = new Kitten[5];} , what is the value of {\tt kittens[0]} ?

\begin{choices}
\correctchoice null
\choice the address of a {\tt Kitten} object
\choice automatically set to 0
\choice undefined
\end{choices}

\newpage

\begin{lstlisting}[language=Java]
public class Doberman {
    private static int dobieCount = 0;
    private String name;

    public Doberman(String name) {
        this.name = name;
        dobieCount++;
    }
    public String reportDobieCount() {
        return name + " says there are " + dobieCount + " dobies.";
    }
    public boolean equals(Doberman other) {
        return this.name.equals(other.name);
    }
}
\end{lstlisting}

\question[3] If no {\tt Doberman} instances have been created, what is true about the following line from another class?
\begin{lstlisting}[language=Java]
        System.out.println("dobieCount: " + Doberman.dobieCount);
\end{lstlisting}
\begin{choices}
\correctchoice It will not compile.
\choice It will compile but will cause a {\tt ClassCastException} at run-time.
\choice It will print ``dobieCount: 0''
\end{choices}

\question[3] What would be printed to the console after executing the following statements?
\begin{lstlisting}[language=Java]
        Doberman fido = new Doberman("Fido");
        Doberman chloe = new Doberman("Chloe");
        System.out.println(chloe.reportDobieCount());
        Doberman prince = new Doberman("Prince");
\end{lstlisting}

\begin{choices}
\choice Chloe says there are 1 dobies.
\correctchoice Chloe says there are 2 dobies.
\choice Chloe says there are 3 dobies.

\end{choices}

\question[3] What would be printed to the console after executing the following statements?
\begin{lstlisting}[language=Java]
        ArrayList daringDobermans = new ArrayList();
        daringDobermans.add(new Doberman("Chloe"));
        System.out.println(daringDobermans.contains(new Doberman("Chloe")));
\end{lstlisting}

\begin{choices}
\choice {\tt true}
\correctchoice {\tt false}
\end{choices}

\question[3] What would be printed to the console after executing the following statements?
\begin{lstlisting}[language=Java]
        ArrayList daringDobermans = new ArrayList();
        Doberman chloe = new Doberman("Chloe");
        daringDobermans.add(chloe);
        System.out.println(daringDobermans.contains(chloe));
\end{lstlisting}

\begin{choices}
\correctchoice {\tt true}
\choice {\tt false}
\end{choices}

\question[3] Given {\tt Doberman chloe = new Doberman("Chloe")}, what would {\tt chloe.toString()} return?

\begin{choices}
\correctchoice Something like ``Doberman@deadbeef''
\choice ``Chloe''
\choice {\tt null}
\end{choices}

\newpage

\begin{lstlisting}
public class Super {
    protected int x = 1;
}
\end{lstlisting}

\begin{lstlisting}
public class Duper extends Super {
    protected int y = 2;

    public Duper(int n) { x += y + n; }

    public String toString() { return new Integer(x).toString(); }
}
\end{lstlisting}

\begin{lstlisting}[numbers=left]
public class Andes {
    static int a = 0;
    static boolean incA() { return ++a > 0; }

    public static void main(String[] args) {
        boolean b = Boolean.parseBoolean(args[0]);
        System.out.println( b && incA() ? new Duper(a) : new Duper(a + 1));
    }
}
\end{lstlisting}

\question[3] What is printed when {\tt java Andes true} is executed on the command line?

\begin{choices}
\choice 3
\correctchoice 4
\choice 5
\end{choices}

\question[3] What is printed when {\tt java Andes false} is executed on the command line?

\begin{choices}
\choice 3
\correctchoice 4
\choice 5
\end{choices}


\vspace{.1in}
\hspace{-.5in}For the next two questions, change line 3 in {\tt Andes.java} to
\begin{lstlisting}
    static boolean incA() { return a++ > 0; }
\end{lstlisting}
\vspace{.1in}

\question[3] What is printed when {\tt java Andes true} is executed on the command line?

\begin{choices}
\choice 3
\choice 4
\correctchoice 5
\end{choices}

\question[3] What is printed when {\tt java Andes false} is executed on the command line?

\begin{choices}
\choice 3
\correctchoice 4
\choice 5
\end{choices}

\question[3] Will the expression {\tt new Duper()} compile?

\begin{choices}
\choice Yes
\correctchoice No
\end{choices}

\newpage

Assume {\tt Trooper} is defined as follows:
\begin{lstlisting}[language=Java]
public class Trooper {
    private String name;
    private boolean mustached;
    public Trooper(String name, boolean hasMustache) {
        this.name = name; this.mustached = hasMustache;
    }
    public String getName() { return name; }
    public boolean hasMustache() { return mustached; }

    public boolean equals(Trooper other) {
        if (this == other) return true;
        if (null == other || !(other instanceof Trooper)) return false;
        Trooper that = (Trooper) other;
        return this.name.equals(that.name) && this.mustached == that.mustached;
    }
    public int hashCode() { return 1; }
}
\end{lstlisting}
And the following has been executed in the same scope as the code in the questions below:
\begin{lstlisting}[language=Java]
  ArrayList<Trooper> troopers = new ArrayList<>();
  troopers.add(new Trooper("Farva", true));
  troopers.add(new Trooper("Farva", true));
  troopers.add(new Trooper("Rabbit", false));
  troopers.add(new Trooper("Mac", true));
\end{lstlisting}



\question[3] What would be the result of the statement   {\tt Collections.sort(troopers)}?

\begin{choices}
\correctchoice The code will not compile.
\choice {\tt troopers} will be sorted in order by name.
\choice {\tt troopers} will be sorted in order by mustache, then name.
\choice {\tt troopers} will not have any duplicate elements.
\end{choices}

\question[3] After executing the statement {\tt Set<Trooper> trooperSet = new HashSet<>(troopers)}, what would be the value of {\tt trooperSet.contains(new Trooper("Mac", true))}?

\begin{choices}
\choice The code will not compile.
\choice {\tt true}
\correctchoice {\tt false}
\choice {\tt void}
\end{choices}

\question[3] Given the definitions of {\tt troopers} and {\tt trooperSet} above, what would {\tt trooperSet.size()} return?

\begin{choices}

\choice 3
\correctchoice 4
\end{choices}

\question[3] After the statement {\tt Set<String> stringSet = new HashSet<>(Arrays.asList("meow", "meow"))} executes, what would be the value of {\tt stringSet.size()}?

\begin{choices}
\correctchoice 1
\choice 2
\end{choices}

\question[3] What would {\tt new Trooper("Ursula", false).equals(new Trooper("Ursula", false)))} return?

\begin{choices}
\correctchoice {\tt true}
\choice {\tt false}
\end{choices}




\newpage





Given the following class definitions:
\begin{lstlisting}
public abstract class Animal {
    public abstract void speak();
    public int legs() { return 4; }
}
\end{lstlisting}

\begin{lstlisting}
public class Mammal extends Animal {
    public void speak() { System.out.println("Hello!"); }
}
\end{lstlisting}

\begin{lstlisting}
public class Canine extends Mammal {
    public void speak() { System.out.println("Grr!"); }
}
\end{lstlisting}


\begin{lstlisting}
public class Dog extends Canine {
    public void speak(String to) { System.out.println("Woof, " + to); }
}
\end{lstlisting}

\begin{lstlisting}
public class Cat extends Mammal {
    public void speak() { System.out.println("Meow!"); }
}
\end{lstlisting}

\question[3] Say we write a subclass of {\tt Mammal} named {\tt Kangaroo} in which we want to override the {\tt legs} method.  Which of the following methods overrides {\tt legs}?
\begin{choices}
\choice  {\tt public void legs() \{ System.out.println(2); \}}
\choice  {\tt public Object legs() \{ return new Integer(2); \}}
\choice  {\tt public double legs() \{ return 2; \}}
\correctchoice None of the above.
\end{choices}

\question[3] Which of the following is an invocation of the method {\tt public void pet(Canine c)}?
\begin{choices}
\correctchoice {\tt pet(new Dog())}
\choice {\tt pet(new Cat())}
\choice {\tt pet(new Mammal())}
\choice {\tt pet(new Animal())}
\end{choices}


\question[3] Assuming {\tt Mammal fido = new Dog();} has been executed, what does {\tt fido.speak()} print?
\begin{choices}
\choice  Hello!
\choice  Woof! Woof!
\choice  Meow!
\correctchoice None of the above.
\end{choices}

\question[3] Assuming {\tt Mammal fido = new Dog();} has been executed, what does {\tt ((Mammal) fido).speak()} print?
\vspace{-.05in}
\begin{choices}
\correctchoice Grr!
\choice  Hello!
\choice  Woof! Woof!
\choice  Meow!
\end{choices}

\question[3] Assuming the statement {\tt Mammal sparky = new Mammal();} has been executed, which of the following statements will compile but cause a {\tt ClassCastException} at run-time?
\begin{choices}
\correctchoice  Dog fido = (Dog) sparky;
\choice Mammal fido = new Dog();
\choice Dog fido2 = (Dog) new Dog();
\choice Cat c = new Dog()
\end{choices}




\newpage

Given the following classes, which have no-arg constructors:

\begin{lstlisting}[language=Java]
public class A extends Throwable { ... }
public class B extends A { ... }
public class C extends RuntimeException { ... }
\end{lstlisting}

\question[3] Which of the following will {\bf not} compile?

\begin{choices}
\choice
\begin{lstlisting}[language=Java]
    A foo(B b) throws C {
        if (true) throw new C();
        return new B();
    }
\end{lstlisting}

\correctchoice
\begin{lstlisting}[language=Java]
    A baz(B b) throws B {
        if (true) throw new A();
        return new B();
    }
\end{lstlisting}

\end{choices}

\question[3] Which of the following will {\bf not} compile?

\begin{choices}
\correctchoice
\begin{lstlisting}[language=Java]
    A foo(B b) throws C {
        if (true) throw new B();
        return new B();
    }
\end{lstlisting}

\choice
\begin{lstlisting}[language=Java]
    A bar(B b) throws C {
        if (true) throw new RuntimeException("c");
        return new B("c");
    }
\end{lstlisting}

\choice
\begin{lstlisting}[language=Java]
    A baz(B b) throws A {
        if (true) throw new A("a");
        return new B("c");
    }
\end{lstlisting}
\end{choices}

\question[3] Given the method signature {\tt A bar(B q) throws C}, will this code
 compile?
\begin{lstlisting}[language=Java]
A m() throws C {
    return bar(new B());
}
\end{lstlisting}

\begin{choices}
\correctchoice Yes
\choice No
\end{choices}

\question[3] Given the method signature {\tt A bar(B q) throws B}, which of the following will {\bf not} compile?

\begin{choices}
\correctchoice
\begin{lstlisting}[language=Java]
A m() throws C {
    return bar(new B());
}
\end{lstlisting}

\choice
\begin{lstlisting}[language=Java]
A m() throws Throwable {
    return bar(new B());
}
\end{lstlisting}

\choice All of the above will compile.
\end{choices}

\question[3] What is the highest superclass of all exceptions?

\begin{choices}
\correctchoice {\tt java.lang.Object}
\choice {\tt java.lang.Throwable}
\choice {\tt java.lang.Exception}
\end{choices}



\newpage


Given the following definitions:

\small

\begin{lstlisting}[language=Java]
public interface Predicate<T> {
    boolean test(T t);
}
\end{lstlisting}

\begin{lstlisting}[language=Java]
    static <E> E find(List<E> es, Predicate<E> p) {
        for (E e: es) if (p.test(e)) return e;
        return null;
    }
\end{lstlisting}

\begin{lstlisting}[language=Java]
public interface Function<T, R> {
    R apply(T t);
}
\end{lstlisting}

\begin{lstlisting}[language=Java]
    static <E, R> List<R> map(List<E> es, Function<E, R> f) {
        List<R> result = new ArrayList<>();
        for (E e: es) result.add(f.apply(e));
        return result;
    }
\end{lstlisting}

and the list:
\begin{lstlisting}[language=Java]
List<String> words = Arrays.asList("Welcome", "To", "Java", "8");
\end{lstlisting}

\normalsize

\question[3] Which of the following expressions would return the first word in {\tt words} that starts with an upper case character?

\begin{choices}
\correctchoice {\tt find(words, s -> Character.isUpperCase(s.charAt(0)))}
\choice {\tt find(map(words, String::split), a -> a[0].isUpperCase())}
\choice {\tt find(words, s -> s.toUpperCase())}
\choice All of the above.
\end{choices}

\question[3] Which of the following expressions would return a list of the lengths of the words in {\tt words}?

\begin{choices}
\choice {\tt map(words, (String s) -> s.length())}
\choice {\tt map(words, String::length)}
\choice {\tt map(map(words, s -> s.split("")), a -> a.length)}
\correctchoice All of the above.
\end{choices}

%% \question[3] What is the type of the following lambda expression?

%% \begin{lstlisting}[language=Java]
%% (String s) -> {System.out.println(s);}
%% \end{lstlisting}

%% \begin{choices}
%% \choice {\tt Function<String>}
%% \choice {\tt Consumer<String>}
%% \choice {\tt Predicate<String>}
%% \correctchoice The answer cannot be determined from the information given.
%% \end{choices}

\question[3] Is {\tt Comparable<T>} a functional interface?

\begin{choices}
\correctchoice Yes
\choice No
\end{choices}


%% \question[3] The time complexity of finding an element in a singly-linked list with only a reference to the first node is \makebox[.25in]{\hrulefill}.
%% \begin{choices}
%% \choice $O(1)$
%% \choice $O(\log n)$
%% \correctchoice $O(n)$
%% \choice $O(n^2)$
%% \choice $O(2^n)$
%% \end{choices}

%% \question[3] The time complexity of adding an element to the front of a singly-linked list with only a reference to the first node is \makebox[.25in]{\hrulefill}.
%% \begin{choices}
%% \correctchoice $O(1)$
%% \choice $O(\log n)$
%% \choice $O(n)$
%% \choice $O(n^2)$
%% \choice $O(2^n)$
%% \end{choices}

%% \question[3] A stack is a \makebox[.25in]{\hrulefill} data structure.

%% \begin{choices}
%% \correctchoice LIFO
%% \choice LIPO
%% \choice FIFO
%% \choice FIDO
%% \end{choices}

%% \question[3] A queue is a \makebox[.25in]{\hrulefill} data structure.

%% \begin{choices}
%% \choice LIFO
%% \choice LIPO
%% \correctchoice FIFO
%% \choice FIDO
%% \end{choices}

%% \question[3] What is true about this code?

%% \begin{lstlisting}
%%     public static int fac(int n) {
%%         return n * fac(n + 1);
%%     }
%%     // ...
%%     int fac5 = fac(5);
%% \end{lstlisting}

%% \begin{choices}
%% \choice Compiles and runs without errors or exceptions.
%% \correctchoice Compiles but program terminates with an error or exception.
%% \end{choices}

\newpage

\question[3] What is true about this code?

\begin{lstlisting}
    public static int fac(int n) {
        if (n >= 1) return 1;
        else return n * fac(n + 1);
    }
    // ...
        int fac5 = fac(5);
\end{lstlisting}

\begin{choices}
\correctchoice Compiles and runs without errors or exceptions.
\choice Compiles but program terminates with an error or exception.
\end{choices}

%% \question[3] Given the following recursive method:
%% \begin{lstlisting}[language=Java]
%%     private static int bs(int[] array, int queryValue, int lo, int hi) {
%%         if (lo > hi) return -1;
%%         int middle = (lo + hi)/2;
%%         if (queryValue == array[middle]) return middle;
%%         else if (queryValue < array[middle]) {
%%             return bs(array, queryValue, middle + 1, hi);
%%         } else {
%%             return bs(array, queryValue, lo, middle - 1);
%%         }
%%     }
%% \end{lstlisting}
%% and an array defined as {\tt int[] a = \{1,2,3,4,5,6,7,8\}}, what would be returned by the call {\tt bs(a, 5, 0, a.length - 1)}? {\bf Examine the code carefully.}

%% \begin{choices}
%% \correctchoice -1
%% \choice 4
%% \choice 5
%% \end{choices}

%% \question[3] What is the worst-case time complexity (Big-O) of the following method?
%% \begin{lstlisting}[language=Java]
%% public int f(int n) {
%%     int s = 0;
%%     for (int i = 0; i < n; i++)
%%         for (int j = 0; j < i; j++)
%%             for (int k = 0; k < j; k++)
%%             s += k;
%%     return s;
%% }
%% \end{lstlisting}

%% \begin{choices}
%% \choice $O(\lg n)$
%% \choice $O(n)$
%% \choice $O(n^2)$
%% \correctchoice $O(n^3)$
%% \end{choices}

%% \question[3] Meow!

%% \begin{choices}
%%   \correctchoice True
%% \end{choices}

\vspace{.1in}

\begin{lstlisting}[language=Java]
    public static int f(int n) {
        if (n < 0) throw new IllegalArgumentException("n < 0");
        if (n <= 1) {
            return n;
        } else {
            return f(n - 1) + f(n - 2);
        }
    }
\end{lstlisting}

\question[3] Given the method {\tt f} above, what is {\tt f(5)}?

\begin{choices}
\choice 0
\choice 4
\correctchoice 5
\choice 120
\end{choices}

%% \question[3] Given the method {\tt f} above, what is {\tt f(4)}?

%% \begin{choices}
%% \choice 0
%% \correctchoice 3
%% \choice 4
%% \choice 24
%% \end{choices}

\vspace{.2in}

\begin{lstlisting}[language=Java]
public class ArrayListQueue<E> {
    private ArrayList<E> elems = new ArrayList<>();
    public void enqueue(E item) {
        ???
    }
    public E dequeue() {
        ???
    }
    public boolean isEmpty() {
        return elems.isEmpty();
    }
}
\end{lstlisting}

\question[3] Given the partial {\tt ArrayListQueue} implementation above, which of the following statements for line 5 would implement {\tt enqueue} in $O(1)$ time?  Do not consider any particular implementation for {\tt dequeue}.

\begin{choices}
\correctchoice {\tt elems.add(item);}
\choice {\tt elems.add(0, item);}
\choice {\tt return elems.remove(elems.size() - 1);}
\end{choices}

\question[3] Given the partial {\tt ArrayListQueue} implementation above, which of the following statements for line 5 would implement {\tt enqueue} in $O(n)$ time? Do not consider any particular implementation for {\tt dequeue}.

\begin{choices}
\choice {\tt elems.add(item);}
\correctchoice {\tt elems.add(0, item);}
\choice {\tt return elems.remove(elems.size() - 1);}
\end{choices}

%% \question[3] Which data structure would you use in an algorithm that tests a string for balanced parentheses of many kinds, e.g., {\tt (), {}, []}?

%% \begin{choices}
%% \choice list
%% \correctchoice stack
%% \choice queue
%% \end{choices}


\end{questions}

\end{document}
